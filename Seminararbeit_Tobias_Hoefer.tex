
\documentclass{article}

\usepackage[utf8]{inputenc} 
\usepackage{lipsum}
\usepackage[margin=1 in,left=1.5in,includefoot]{geometry}
\usepackage{graphicx}
\usepackage{float}
\usepackage[hidelinks]{hyperref} %Allows for clickable references

%Header and Footer Stuff
\usepackage{fancyhdr}
\pagestyle{fancy}
\fancyhead{}
\fancyfoot{}
\fancyfoot[R]{ \thepage\ }
\renewcommand{\headrulewidth}{0pt}
\renewcommand{\footrulewidth}{0pt}
%

\begin{document}

\begin{titlepage}
	\begin{center}
	\line(1,0){330} \\
	\vspace{.75 cm}
	\huge{\bfseries Vektorräume und Hilberträume}\\
	\vspace{.25 cm}
	\line(1,0){330} \\
	\vspace{1.5 cm}
	\textsc{\LARGE Hochschule M\"unchen}\\
	%Figure
	%\begin{figure}[H]
	%	\centering
%		\includegraphics[height=0 cm] {path}
	%	\caption[Optional caption]{Real, local caption}
	%	\label{fig:tobias}
	%\end{figure}
	\vspace{10 cm}
	\end{center}
	\begin{flushright}
	\textsc{\large Tobias H\"ofer\\26.12.2016}
	\end{flushright}
\end{titlepage}

\tableofcontents
\thispagestyle{empty}
\cleardoublepage

\setcounter{page}{1}
\section{Einleitung}\label{sec:intro}
\lipsum[1]
\section{Vektorraum}\label{sec:vec}
\subsection{Definition}\label{sec:vecdef} \cite{ref:one}
\section{Hilbertraum}\label{sec:hil}
\subsection{Definition}\label{sec:hildef}
\section{Hilbertraum in der Quanteninformatik}\label{sec:def}
\section{Zusammenfassung}\label{sec:intro}

%References
\cleardoublepage
\bibliographystyle{plain}
\bibliography{/Users/tobiashofer/Seminararbeit/references/firstref.bib}

\end{document}