
\documentclass{article}

\usepackage[utf8]{inputenc} 
\usepackage{lipsum}
\usepackage[margin=1 in,left=1.5in,includefoot]{geometry}
\usepackage{graphicx}
\usepackage{float}
\usepackage[hidelinks]{hyperref} %Allows for clickable references

%Header and Footer Stuff
\usepackage{fancyhdr}
\pagestyle{fancy}
\fancyhead{}
\fancyfoot{}
\fancyfoot[R]{ \thepage\ }
\renewcommand{\headrulewidth}{0pt}
\renewcommand{\footrulewidth}{0pt}
%




\begin{document}

\begin{titlepage}
	\begin{center}
	\line(1,0){330} \\
	\vspace{.75 cm}
	\huge{\bfseries D-Wave-Computer}\\
	\vspace{.25 cm}
	\line(1,0){330} \\
	\vspace{1.5 cm}
	\textsc{\LARGE Hochschule M\"unchen}\\
	%Figure
	%\begin{figure}[H]
	%	\centering
%		\includegraphics[height=0 cm] {path}
	%	\caption[Optional caption]{Real, local caption}
	%	\label{fig:Stefan}
	%\end{figure}
	\vspace{10 cm}
	\end{center}
	\begin{flushright}
	\textsc{\large Stefan Ronczka\\26.12.2016}
	\end{flushright}
\end{titlepage}

\tableofcontents
\thispagestyle{empty}
\cleardoublepage

\setcounter{page}{1}
\section{Einleitung}\label{sec:intro}

Wozu braucht die Welt Quantencomputer. Diese Frage Lässt sich beantworten durch einen Einfaches Beispiel 
Wenn man sich ein wenig mit Quantum Computing beschäftigt hat fällt einem relativ schnell auf, dass man zwar Algorithmen mit Hilfe der Hadamar Matrix oder Hadamar Gatter
\section{Der erste Anlauf für einen Quantencomputer}\label{sec:historyDWave}
\subsection{Aufbau}\label{sec:funktion}\cite{ref:one}
\section{Quantum annealing}\label{sec:baseAlgo}
\subsection{Schwierige Probleme lösen mit Physik}\label{sec:historyAlgo}
\subsection{Der Algorithmus}\label{sec:algo}
\section{Struktur der Software}\label{sec:software}
\section{Anwendung auf komplexe Probleme}\label{sec:problems}

%References
\cleardoublepage
\bibliographystyle{plain}
\bibliography{/references/firstref.bib}





\end{document}